\documentclass{Physics_Report}
%% ---- information --- %%
\newcommand{\department}{物理科学与技术}
\newcommand{\major}{物理学}
\newcommand{\Year}{2021}
\newcommand{\Month}{3}
\newcommand{\Date}{16}
\newcommand{\exptitle}{实验名称}
\newcommand{\stuid}{2018xxxxxxxxx}
\newcommand{\name}{姓名名}
\newcommand{\grade}{18级}



\begin{document}
%% ---- Title ---- %%
\begin{center}
	\zihao{2}\kaishu 武\ \ 汉\ \ 大\ \ 学\ \ 物\ \ 理\ \ 科\ \ 学\ \ 与\ \ 技\ \ 术\ \ 学\ \ 院\\
	物\ \ 理\ \ 实\ \ 验\ \ 报\ \ 告\\
	\zihao{6}~\\
	
\end{center}
\song \zihao{4}\underline{\department} 学院 \qquad\qquad \underline{\major} 专业\qquad\qquad \underline{\Year}\ 年 \underline{\Month}\ 月 \underline{\Date}\ 日\\
实验名称 \ \ \underline{\exptitle}\\
姓名 \ \ \underline{\name} \qquad 年级\ \ \underline{\grade} \qquad 学号\ \ \underline{\stuid}\qquad  成绩\ \ \_\_\_\_\_\_
\begin{multicols}{2}
\noindent 实验报告内容:\\
\zihao{-4}
 一、实验目的\\二、主要实验仪器\\三、实验原理\\四、实验内容与步骤~\\\\五、数据表格\\六、数据处理及结果表达\\七、实验结果分析\\八、习题
\end{multicols}

%% ---- Body Part ---- %%
\zihao{-4}
\section{实验目的}
\begin{enumerate}
	\item 培养学生xxx的能力
	\item 练习使用xxx的仪器
	\item 深入理解xxx的概念
\end{enumerate}

\section{主要实验仪器}
\subsection{仪器一:xxx}
仪器一由xxx和xxx组成,其结构如图\ref{yiqi}所示。
\begin{figure}[htbp]
	\centering
	\includegraphics[width=.4\textwidth]{yiqi.png}
	\caption{实验仪器一}
	\label{yiqi}
\end{figure}
\subsection{仪器二:xxx}
仪器二由以下几个部分组成。
\subsubsection{结构一:xxx}

\subsubsection{结构二:xxx}

\section{实验原理}
\zhlipsum*[1]%随机生成文本
\section{实验内容与步骤}
\subsection{步骤一}
\zhlipsum*[2]%随机生成文本
\subsection{步骤二}
\zhlipsum*[3]%随机生成文本

\section{数据表格}
实验得到的数据如下表\ref{shuju}所示。
\begin{table}[htbp]
	\centering
	\caption{实验数据}
	\begin{tabular}{c|c}
		 \hline
		 \makebox[0.3\textwidth][c]{符号}	&  \makebox[0.4\textwidth][c]{测量结果} \\ \hline
		 xxx&xxx\\
		 xxx&xxx\\
		 xxx&xxx\\
		 xxx&xxx\\
		 xxx&xxx\\
		 xxx&xxx\\
		 \hline
	\end{tabular}
	\label{shuju}
\end{table}

\section{数据处理及结果表达}
\zhlipsum*[2-3]
$$ x+y = a + b$$

\begin{equation}
	\delta  = \left| {\frac{{\bar e - {e_t}}}{{{e_t}}}} \right|=0.9988\%
\end{equation}
\section{实验结果分析}
\subsection{实验结果陈述}
实验求得xxx验证了xxxx。在误差允许的范围内测出了xxx。
\subsection{实验误差分析}
实验的结果存在误差,误差的可能来源有:
\begin{itemize}
	\item 仪器老旧
	\item xxx操作不准确
	\item 其他原因
\end{itemize}
\section{习题}
\begin{enumerate}
	\item xxxxxxxx
	
	答:
	
	\item xxxxxxxx
	
	答:
\end{enumerate}
~\\
~\\
\fbox{\shortstack[l]{教\\\ \\\ \\师\\\ \\评\\\ \\语\qquad \qquad \qquad \qquad \quad \qquad \qquad \qquad \qquad 指导教师:\qquad\qquad \quad \Year 年 \Month 月 \Date 日\qquad \ }}
\end{document} 